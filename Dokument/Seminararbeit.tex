\documentclass[conference]{IEEEtran}
\IEEEoverridecommandlockouts
% The preceding line is only needed to identify funding in the first footnote. If that is unneeded, please comment it out.
\usepackage{cite}
\usepackage{amsmath,amssymb,amsfonts}
\usepackage{algorithmic}
\usepackage{graphicx}
\usepackage{textcomp}
\usepackage{xcolor}
\def\BibTeX{{\rm B\kern-.05em{\sc i\kern-.025em b}\kern-.08em
    T\kern-.1667em\lower.7ex\hbox{E}\kern-.125emX}}
    
    
\begin{document}


\title{Implementierung Objektorientierter-Konstrukte in der Java Virtual Machine\\}

\author{\IEEEauthorblockN{Braun Georg}
\IEEEauthorblockA{\textit{Fachhochschule Aachen} \\
Aachen, Deutschland \\
Georg.Braun@alumni.fh-aachen.de}
}

\maketitle

\begin{abstract}

\end{abstract}



\section{Einleitung}
Java ist einer der beliebtesten objektorientierten Programmiersprachen und ist schon seit einigen Jahren eine etabliert. Neben Java gibt es noch diverse andere Programmiersprachen, wie zum Beispiel Kotlin oder Clojure die alle eine elementare Gemeinsamkeit haben. Alle diese Sprachen werden nicht speziell für eine Rechnerarchitektur, sondern in einen Bytecode kompiliert. Dieser Bytecode kann von Java Virtual Machines verarbeitet werden. Somit ist eine Ausführung des Code unabhängig von der zugrunde liegenden Rechnerarchitektur möglich (vorausgesetzt die Java Virtual Machine kann auf dieser betrieben werden). Im Fokus dieses Dokuments steht die Bytecode Repräsentation von objektorientierten Konstrukten. Dazu wird zunächst auf die Struktur der Java class Dateien eingegangen um darauf die zum Verständnis notwendigen Grundlagen der Struktur der Java Virtual Machine aufzugreifen. Danach findet eine Erklärung statt wie auf Bytecode-Ebene Methoden aufgerufen werden. Neben der Umsetzung von Konstruktor-Aufrufen werden auch die Zugriffe auf Objektfelder und Klassen- bzw. Static-Felder erläutert. Zuletzt wird noch auf die Bytecode-Implementation von Typ-Überprüfungen eingegangen.  

\section{Aufbau der class Datei}

\section{Java Virtual Machine Grundlagen}

\section{Methodenaufrufe}
\subsection{Ablauf eines Methodenaufrufs}

\subsection{Vergleich der Methodenaufrufe}
\subsubsection{Dynamic linking}
\subsubsection{Klassen-/Static-Methoden}
\subsubsection{Instanz-Methoden}

\subsection{Beispielprogramm}

\section{Konstruktor}

\section{Zugriff auf Felder}
\subsection{Zugriff auf Objekt-Felder}
\subsection{Zugriff auf Klassen-/Static-Felder}

\section{Typüberprüfung}

\bibliography{literatur.bib} 

\end{document}
